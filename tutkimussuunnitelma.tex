\documentclass[12pt,a4paper,finnish,oneside]{article}

% Valitse 'input encoding':
%\usepackage[latin1]{inputenc} % merkistökoodaus, jos ISO-LATIN-1:tä.
\usepackage[utf8]{inputenc}   % merkistökoodaus, jos käytetään UTF8:a
% Valitse 'output/font encoding':
%\usepackage[T1]{fontenc}      % korjaa ääkkösten tavutusta, bittikarttana
\usepackage{ae,aecompl}       % ed. lis. vektorigrafiikkana bittikartan sijasta
% Kieli- ja tavutuspaketit:
\usepackage[finnish]{babel}
% Muita paketteja:
% \usepackage{amsmath}   % matematiikkaa
\usepackage{url}       % \url{...}
% Tästä saa compactitemin
\usepackage{paralist}

% Kappaleiden erottaminen ja sisennys
\parskip 1ex
\parindent 0pt
\evensidemargin 0mm
\oddsidemargin 0mm
\textwidth 159.2mm
\topmargin 0mm
\headheight 0mm
\headsep 0mm
\textheight 246.2mm

\pagestyle{plain}

% ---------------------------------------------------------------------

\begin{document}

% Otsikkotiedot: muokkaa tähän omat tietosi

\title{TIK.kand tutkimussuunnitelma:\\[5mm]
Videokoodauksen rinnakkaistaminen}

\author{Miro Nurmela, 84745F\\
Aalto-yliopisto,\\
\url{miro.nurmela@aalto.fi}}

\date{\today}

\maketitle

% ---------------------------------------------------------------------

\vspace{10mm}

% MUOKKAA TÄHÄN. Jos tarvitset tähän viitteitä, käytä
% tässä dokumentissa numeroviitejärjestelmää komennolla \cite{kahva}.
%
% Paljon kandidaatintöitä ohjanneen Vesa Hirvisalon tarjoama 
% sabluuna. Kursivoidut osat \emph{...} ovat kurssin henkilökunnan
% lisäämiä. 

\textbf{Kandidaatintyön nimi:} Videokoodauksen rinnakkaistaminen

\textbf{Työn tekijä:} Miro Nurmela

\textbf{Ohjaaja:} Vesa Hirvisalo


\section{Tiivistelmä tutkimuksesta}

Videodatan määrä on rajussa kasvussa monilla sovellusalueilla \cite{cisco}
\cite{youtube}. Datan määrän kasvaessa tarvitaan tehokkaampia menetelmiä
datan koodaukseen, mutta  Mooren lain mukainen eksponentiaalinen laskentatehon
kasvu ei voi jatkua ikuisesti \cite{sutter}. Laskentaintensiivisiin tehtäviin
tullaan siis jatkossa tarvitsemaan rinnakkaislaskentaa.

Tämän työ esittelee rinnakkaisohjelmoinnin ja videokoodauksen perusperiaatteet
sekä rinnakkaisohjelmoinnin soveltamista videokoodaukseen.

\section{Tavoitteet ja näkökulmat}

Rinnakkaisuuden hyödyt ovat selvät - moniydin ja -prosessorijärjestelmissä
kannattaa käyttää mahdollisimman suuri osa ytimistä ja prosessoreista
mahdollisimman suorituskykyisten ohjelmistojen rakentamiseksi. Työn tavoitteena on
osoittaa, että rinnakkaisista ratkaisusta on hyötyä videokoodausohjelmistoja
toteutettaessa.

Keskeisiä kysymyksiä ovat siis:
\begin{compactitem}
\item Miten videokoodaustekniikat rinnakkaistuvat?
\item Kuinka suuri hyöty rinnakkaistamisesta on suorituskyvyn kannalta?
\item Miten erilaiset videokoodaustekniikat soveltuvat erilaisille 
kiihdytysalustoille?
\end{compactitem}

Työn keskiössä ovat rinnakkaiset ohjelmistoratkaisut sekä niiden suorituskyky. Seuraako rinnakkaisista
ratkaisuista tehokkaampia (nopeampia) videokoodausmenetelmiä? Millaisia
keinoja toteuttaa rinnakkaista laskentaa on ja millaisia ratkaisuja se vaatii
sovelluksilta ja laskentayksiköiltä?

Arvosanan suhteen tavoitteena on tehdä kiitettävä työ.

\section{Tutkimusmateriaali}

Materiaalia rinnakkaisohjelmoinnista, videokoodauksista ja rinnakkaisesta
videokoodauksesta on runsaasti. Tarjolla on kirjoja, väitöskirjoja ja muita
opinnäytetöitä, konferensseja, workshopeja ja papereita aiheeseen liittyen.
Monipuolisen tutkimuksen lisäksi erilaisten videokoodausstandardien dokumentaatiot
tarjoavat hyvä kuvan menetelmien toiminnasta. Yleisvaikutelmia erilaisista
ratkaisuista voi löytyä myös teknisissä yleislehdissä, joskaan ei-tieteellisinä
lähteinä näitä ei primäärilähteinä voi käyttää. Tavoitteiden valossa
materiaalia kattavaan työhön on aivan riittävästi.

Materiaalista on ensivaikutelman perusteella runsauden pula.
Hankittuani perustiedot videokoodauksista ja rinnakkaisohjelmoinnista
voinen keskittyä rinnakkaisesta videokoodauksesta tehtyyn tutkimukseen.
Aika-arviona perustietojen lähteisiin tutustumiseen ja tiedon omaksumiseen
voisi kulua n. 10 tuntia ja varsinaisiin tutkimuslähteisiin n. 30 tuntia.
Näiden lomassa tapahtuisi toki jo kirjoittamista.

\section{Tutkimusmenetelmät}

Tämä tutkimus on kirjallisuustutkimus, jonka lähteet tulevat olemaan kirjoja,
tieteellisiä artikkeleita, väitöskirjoja, opinnäytetöitä ja mahdollisia
epätietellisiä lähteitä (ei-tieteelliset julkaisut, internet).
Joitakin kirjoja on tarjolla fyysisinä kappaleina kirjastoissa, kun taas
toisia on tarjolla sähköisesti. Sama pätee artikkeleihin, joskin suurin osa
niistä on verkossa. Internetin lähteisiin pitää suhtautua suurimmalla
lähdekritiikillä. Tuorein tieto saattaa kuitenkin sijaita erilaisissa blogeissa
tai muissa verkkojulkaisuissa.

Minulle sopivin tapa työskennellä on ottaa ensin selvää perusasioista (tässä
tapauksessa videokoodausmenetelmistä ja rinnakkaisohjelmoinnista) ja siirtyä
sen jälkeen sovellettuun sisältöön (rinnakkaiset videokoodausmenetelmät).

Työn kulku tulee siis olemaan seuraavanlainen:
\begin{compactitem}
\item Lähteiden kokoaminen (lähderyhmien valinta, lähteiden arviointi ja
lukeminen)
\item Tiedonhankinta perusasioista (tiedon organisointi)
\item Perusasioista kirjoittaminen (raportointi)
\item Tiedonhankinta sovelletusta asiasta (tiedon organisointi)
\item Sovelletun osan kirjoittaminen (raportointi)
\item Yhteenveto (raportointi)
\item Nyt kun suurin työ on tehty, kirjoitetaan työhön johdanto, jossa
käsitellään tarvittavat asiat
\item Hiotaan
\end{compactitem}

\section{Haasteet}

Haasteena tulee luultavasti olemaan laajasta materiaalista napakan työn
tekeminen. Olisi helppoa kahmia suuri määrä lähteitä ja kirjoitella
niistä pintapuolisia havaintoja. On olennaista löytää riittävän tarkka rajaus
varsinkin laskentapuolesta puhuttaessa. Olisi myös helppoa mennä liian
syvälle yksittäisiin videokoodaustekniikoihin, joten täytyy myös löytää
sopiva taso yleisten menetelmien ja yksityiskohtaisten kuvausten väliltä.

Oikeiden abstraktiotasojen löytämisen lisäksi ainainen haaste on ajanhallinta.
Minulla ei ole erityisen montaa kurssia keväällä käynnissä, mutta kilpaurheilu,
sosiaaliset velvoitteet ja muut opiskelun ulkopuoliset asiat saattavat haitata
työntekoa. Tavoite on kuitenkin pitää häiriöt hallittavissa.

\section{Resurssit}

Työn tekijänä toimii allekirjoittanut ja ohjaajana Vesa Hirvisalo. Aikaa on
käytössä normaalin opiskelun ohessa riittävästi.

Varsinaiseen kandintyöhön ei tule kokeellista osaa, sillä työn on tarkoitus
toimia napakkana kirjallisuustutkimuksena. Jos työ sujuu hyvin, niin sitä
saattaa seurata erikoistyönä toteutettu kokeellinen osuus.

\section{Aikataulu}

\begin{tabular}{|p{30mm}|p{120mm}|}
\hline
vko 4-5   & Lähteiden keruu ja luku, tutkimussuunnitelman palautus \\ \hline
vko 6-7   & Leipätekstin kirjoitus \\ \hline
vko 8-9   & Johtopäätökset, johdannot ja muut osiot \\ \hline
vko 10-11 & Hiontaa ja tarkastusta \\ \hline
vko 12    & Palautuskelpoinen työ \\ \hline
vko 12 - 23 & Presentaation valmistelu tällä aikavälillä \\ \hline
\end{tabular}

Muut päivämäärät, kuten palautteiden dl:t, löytyvät Nopasta.

Aikataulu on tiivis ja etupainotteinen kolmesta syystä. Kun työn saa käyntiin,
ei ole syytä aikailla ja toisaalta etupainoisuus antaa liikkumavaraa mahdollisia
ongelmia kohdatessa. Toisekseen jos työ valmistuu yllä olevan aikataulun
puitteissa, jää aikaa soveltavan työn tekemiseen.

\section{Esittäminen}

Alla on esitetty alustava runko tutkielman pohjaksi.

\begin{enumerate}
	\item Johdanto
	\item Aiempi tutkimus ja taustaa (nimi päätetään myöhemmin)
	\item Videokoodaus
		\begin{enumerate}
			\item Videokoodauksen perusteet
			\item Videokoodauksen tarpeellisuus
		\end{enumerate}
	\item Rinnakkaislaskenta
		\begin{enumerate}
			\item Rinnakkaislaskennan peruskäsitteet
			\item Rinnakkaislaskenta ja nykypäivän tietokonejärjestelmät
		\end{enumerate}
	\item Videokoodaus ja rinnakkaislaskenta
		\begin{enumerate}
			\item Rinnakkaisuuden hyödyt videokoodaukselle
			\item Erilaiset kiihdytysalustat ja ohjelmistoratkaisut
		\end{enumerate}
	\item{Yhteenveto}
\end{enumerate}

% ---------------------------------------------------------------------
%
% ÄLÄ MUUTA MITÄÄN TÄÄLTÄ LOPUSTA

% Tässä on käytetty siis numeroviittausjärjestelmää. 
% Toinen hyvin yleinen malli on nimi-vuosi-viittaus.

% \bibliographystyle{plainnat}
\bibliographystyle{finplain}  % suomi
%\bibliographystyle{plain}    % englanti
% Lisää mm. http://amath.colorado.edu/documentation/LaTeX/reference/faq/bibstyles.pdf

% Muutetaan otsikko "Kirjallisuutta" -> "Lähteet"
\renewcommand{\refname}{Lähteet}  % article-tyyppisen

% Määritä bib-tiedoston nimi tähän (eli lahteet.bib ilman bib)
\bibliography{lahteet}

% ---------------------------------------------------------------------

\end{document}

