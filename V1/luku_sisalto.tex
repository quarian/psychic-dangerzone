% --------------------------------------------------------------------

\section{Johdanto}

\textbf{DISCLAIMER:} Huonoa, muistiinpanonomaista tekstiä. Tullaan parantamaan.

\newpage

\section{Aiempi tutkimus ja taustaa}

\newpage

\section{Videokoodaus}

Viimeisen viidentoista vuoden aikana suurin osa videodatasta on muuttunut
analogisesta digitaaliseksi. VHS-kaseteista ja analogisista TV-lähetyksistä
on siirrytty Blu-Ray -tekniikoihin, kännykkäkameroihin
teräväpiirtotelevisioihin. Tekniikan kehitys ei näy ainoastaan tallennusmedioissa,
sillä niin ikään tiedonsiirto on kehittynyt ja muuttunut monin paikoin
langattomaksi.(\cite{h264})
Videodatan määrä on myös valtavassa kasvussa (\cite{cisco}, \cite{youtube}).
Raaka videodata suuret määrät tallennustilaa eikä sen siirtäminen
langattomasti ole mahdollista - tarvitaan siis teknologia, jolla videodataa pakataan ja puretaan
(enkoodataan ja dekoodataan). Tätä pakkaamisen ja purkamisen prosessia
kutsutaan videokoodaukseksi. Pakkaamisen ja purkamisen lisäksi videokoodaus kattaa
myös signaalin reaaliaikaisesta kääntämisestä (transkoodaus). Transkoodaus
tarkoittaa esimerkiksi enkoodatun datan kääntämistä toiseen koodausstandardiin (\cite{mpeg_app}).
Seuraavissa alaluvuissa käsitellään videokoodausmenelmien perusteita. Tästä lähin näihin viitataan termein
enkoodaus, dekoodaus ja transkoodaus.

CODEC (COder DECOder pair) viittaa ohjelmistoon tai laitteeseen, joka suorittaa
videokoodausta (\cite{h264}).


\subsection{Videokoodauksen peruskäsitteet}

\subsubsection{Videodatan matka lähteestä näyttölaitteelle}

Videodata saa alkunsa lähteestä, joka on tyypillisesti kamera, joka tallentaa
raakadatan. Tämän jälkeen suoritetaan enkoodaus, jonka jälkeen tiivistetty data
voidaan tallentaa tai siirtää. Tosimaailma on ajallisesta ja paikallisesti jatkuvaa,
mutta digitaalinen maailma ei. Jotta videodataa voidaan tallentaa, täytyy
maailmasta ottaa näytteitä eri ajanhetkillä. Tyypillisesti otokset eri ajanhetkiltä
ovat nelikulmaisia ruudukkoja - tällainen diskreetti esitys voidaan enkoodata
ja esimerkiksi kameran muistikortille tallentaa. Näin peräkkäisistä tallennetuista
ruudukoista voidaan koota myöhemmin videokuvaa.(\cite{h264})

Jos enkoodaus on näyttölaitteelle väärässä
formaatissa, täytyy suorittaa transkoodaus oikeen formaattiin.

Prosessin toisessa päässä tiivistety data dekoodataan ja järjestetään niin, että
se voidaan halutulla kohdelaitteella näyttää käyttäjälle.

\subsubsection{Videouvan esittäminen ja tallentaminen, diskreetti kosinimuunnos}

Kuten mainittu, digitaalinen maailma ei todellisen maailman tavoin ole jatkuva.
Videokuvaa tallentaessa täytyy maailmasta tellennettu kuva sovittaa haluttuun
formaattin - suorakaiteen muotoiseen ruudukkoon näytteitä maailmasta.

Digitaaliset kuvat esitetään siis näytteinä tosimaailmasta - näytteet tunnetaan
paremmin kuvapisteinä eli pikseleinä.
Näistä kuvapisteistä kootaan peräkkäisiä ruutuja, joita peräkkäin toistamalla
saadaan aikaan vaikutelma elävästä kuvasta.
Värikuvaa kuvatessa jokaisella kuvapisteellä täytyy olla tieto siinä esiintyvien
värien määrästä. Suositussa RGB-väriavaruudessa jokaisella kuvapisteellä on
kolme parametriä - punaisen, vihreän ja sinisen värin määrä. Tiheys, jolla
ruutuja tallennetaan määrää ruutunopeuden (frame rate). Erään mittarin
videokuvan laadusta antavat kuvapisteiden määrä ja ruutunopeus. (\cite{h264})

Videokoodausta varten on olemassa häviöllisiä ja häviöttömiä metodeita.
Yhteistä niille on videodatan vahvan spatiaalisen ja ajallisen redundanssin
hyödyntäminen. Käytännössä peräkkäisissä ruuduissa on suurella
todennäköisyydellä lähes samat kuvapisteet ja yhdessä ruudussa lähekkäiset
kuvapisteet muistuttavat toisiaan suurella todennäköisyydellä. (\cite{h264})


Raakoja näytteitä ei ole tapana säilyttää, vaan ne muunnetaan paremmin
enkoodaukseen sopivaan muotoon. Esitellään tässä yksi yleisimmistä
muunnoksista, diskreetti kosinimuunnos (DCT).

DCT on Fourier-muunnoksen kaltainen muunnos, joka muuntaa videokoodauksen
tapauksessa ruutujen näyteblokkien pikseliarvoja eri taajuuksilla värähteleviksi
kosinifunkitoiksi. Saavuettu etu on se, että häviöllisissä koodausmenetelmissä
voidaan jättää koodamatta pienet korkeataajuiset funktiot - niitä ihminen
on huono huomaamaan. (\cite{h264})


\newpage

\subsection{Videokoodauksen tarpeellisuus}

\newpage

\section{Rinnakkaislaskenta}

Rinnakkaislaskennan (parallel computing) tavoite on parantaa laskennan
suorituskykyä. Käsitteellisesti rinnakkaislaskentaa ei kannata sekoittaa
samanaikaiseen laskentaan (concurrent computing). Akateemisessa mielessä
jälkimmäinen tarkoittaa rinnakkaisen laskennan oikeellisuuteen, kun taas
ensimmäinen keskittyy saamaan hajautetusta ja rinnakkaisesta laskennasta
näkyviä hyötyjä. Tämä työ keskittyy rinnakkaisesta laskennasta saatuihin
suorituskyvyn parannuksiin. (citation needed)

\subsection{Rinnakkaisuuden peruskäsitteet}

Rinnakkaisuutta voidaan jakaa kahteen osaan sen perusteella, mihin rinnakkaisuus
kohdistuu. Tehtävärinnakkaisessa laskennassa rinnakkain suoritetaan erilaisia
tehtäviä - esimerkiksi saman ohjelman eri säikeitä. Datarinnakkaisessa
ohjelminnissa taas rinnakkaisuus syntyy siitä, että operoidaan samaan aikaan
eri datan osa-alueilla. Datarinnakkainen lähestymistapa sopii varsin hyvin
videokoodauksen rinnakkaistamiseen - jos esimerkiksi tehdään DCT, voisaan
kukin blokki käsitellä erikseen (blokit riippumattomia toisistaan), joten
saavuetyt hyödyt laskenta-ajassa ovat huomattavat. (citation needed)

\begin{compactitem}
	\item asiaa race conditioneista, mutexeista, synkronoinnista?
	\item amdahl?
\end{compactitem}

\subsection{Rinnakkaislaskenta ja nykypäivän tietokonejärjestelmät}

Nykypäivänä suurin osa laitteista, jotka pystyvät toistamaan videota ovat
moniytimisä tietokoneita - jopa joissakin televisioissa on nykykän
kaksoisydinprosessoireita. Samalla prosessorit ovat kehittyneet suorittamaan
useampia käskyjä kerrallaan (pipelining), joten rinnakkaisuus on ehdottomasti
nykypäivää. (citation needed)

\newpage

\section{Videokoodaus ja rinnakkaislaskenta}

\newpage

\subsection{Rinnekkaisuuden hyödyt videokoodaukselle}

\newpage

\subsection{Erilaiset kiihdytysalustat ja ohjelmistoratkaisut}

\newpage

\section{Yhteenveto}


% --------------------------------------------------------------------

