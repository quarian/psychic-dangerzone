% Tiivistelmät tehdään viimeiseksi. 
%
% Tiivistelmä kirjoitetaan käytetyllä kielellä (JOKO suomi TAI ruotsi)
% ja HALUTESSASI myös samansisältöisenä englanniksi.
%
% Avainsanojen lista pitää merkitä main.tex-tiedoston kohtaan \KEYWORDS.

%  Tiivistelmä on muusta työstä täysin irrallinen teksti, joka
%  kirjoitetaan tiivistelmälomakkeelle vasta, kun koko työ on
%  valmis. Se on suppea ja itsenäinen teksti, joka kuvaa olennaisen
%  opinnäytteen sisällöstä. Tavoitteena selvittää työn merkitys
%  lukijalle ja antaa yleiskuva työstä. Tiivistelmä markkinoi työtäsi
%  potentiaalisille lukijoille, siksi tutkimusongelman ja tärkeimmät
%  tulokset kannattaa kertoa selkeästi ja napakasti. Tiivistelmä
%  kirjoitetaan hieman yleistajuisemmin kuin itse työ, koska teksti
%  palvelee tiedonvälitystarkoituksessa laajaa yleisöä.
%
%  Tiivistelmän rakenne: 
%teksti jäsennetään kappaleisiin (3--5 kappaletta);
%ei väliotsikkoja; 
%ei mitään työn ulkopuolelta; 
%ei tekstiviitteitä tai lainauksia;
%vähän tai ei ollenkaan viittauksia työhön 
%(ei ollenkaan: ``luvussa 3'' tms., mutta koko työhön voi 
%viitata esim. sanalla ``kandidaatintyössä'';
%ei kuvia ja taulukoita.
%
%Tiivistelmässä otetaan ``löysät pois'':
%ei työn rakenteen esittelyä;
%ei itsestäänselvyyksiä;
%ei turhaa toistoa;
%älä jätä lukijaa nälkäiseksi, eli kerro asiasisältö, 
%älä vihjaa, että työssä kerrotaan se.
%
%Tiivistelmän tyypillinen rakenne: 
%(1) aihe, tavoite ja rajaus 
%(heti alkuun, selkeästi ja napakasti, ei johdattelua);
%(2) aineisto ja menetelmät (erittäin lyhyesti);
%(3) tulokset (tälle enemmän painoarvoa); 
%(4) johtopäätökset (tälle enemmän painoarvoa).

\begin{fiabstract}

Tämä kandinaatintyö on kirjallisuustutkimus rinnakkaislaskennan soveltamisesta
videokoodaukseen. Työ esittelee videokoodauksen ja rinnakkaislaskennan
perusteita sekä käytännön sovelluksia rinnakkaisista videokoodausmenetelmistä.
Tavoite on löytää videokoodausta tehostavia rinnakkaisia ratkaisuja. Työssä
ei keskitytä syvällisesti esimerkiksi rinnakkaisiin laitteisto- tai
ohjelmistoratkaisuihin.

Tutkimuksen perusteella olemassa olevat videokoodausmenetelmät nojaavat
jäykkään laskentaan, joka perustuu vahvasti videodatan sisäisiin riippuvuuksiin
ja niitä hyödyntäviin optimointeihin. Videodatan laadun parantuessa tällaiset
menetelmät ovat riittävän tehokkaita ainoastaan tehokkaimmilla yksiytimisillä
prosessoreilla. Olemassa olevia menetelmiä ei ole suunniteltu rinnakkaisuutta
silmällä pitäen, joten niiden rinnakkaistaminen on vaikeaa.

Rinnakkaislaskenta tarjoaa yksiytimisiä prosessoreita paremman suorituskyvyn,
mutta tutkimuksen perusteella rinnakkaislaskentaan liittyy paljon haasteita.
Rinnakkaislaskennan skaalautuvuus on haaste algoritmien suunnittelulle.
Rinnakkaisuus tuo laskentaan kustannuksia, kuten laskentayksiköiden välistä
viestintää, muistinhallintaa ja synkronointia, joita ei tavallisessa
laskennassa kohdata. Yleispäteviä ratkaisuja näihin ongelmiin ei ole löydetty,
sillä rinnakkaislaskennan kenttä on laaja, kiihdytysalustat monimuotoiset ja
tekniikka jatkuvasti kehittyvää.

Haasteista huolimatta tutkimuksessa esitellään onnistuneita rinnakkaisia
videokoodaustoteutuksia. Videodatan riippuvuuksien purkaminen, menetelmien
optimointi ja uudet standardit mahdollistavat entistä tehokkaampaa ja
parempilaatuista videokoodausta. Tutkimuksen johtopäätöksenä helppo 
rinnakkaistuvuus tulisi ottaa tulevissa standardeissa ja
videokoodausmenetelmissä yhdeksi päätavoitteista.

\begin{comment}
Tämä kandinaatintyö käsittelee videokoodausmenetelmien rinnakkaislaskennallisia
toteutuksia. Työn tavoitteena on osoittaa, että rinnakkaislaskentaa
hyödyntämällä voidaan toteuttaa perinteisiä menetelmiä tehokkaampaa videon
pakkaamista ja purkamista rinnakkaisohjelmoinnin ongelmista huolimatta.
Käsittely keskittyy rinnakkaislaskennasta saatavaan
laskentatehon kasvuun eikä esimerkiksi rinnakkaisen laskennan oikeellisuuteen
tai sen järjestämiseen. Videokoodausmenetelmiä käsitellään hyvin yleisellä
tasolla menemättä yksittäisten standardien toteutuksiin tarkemmin.

Tutkimus on kirjallisuustutkimus jonka lähteinä on ollut kokoelma artikkeleita,
kirjoja ja verkkolähteitä.

Tutkimuksen tulos on, että rinnakkaislaskenta tehostaa videokoodausprosessia.
Tärkeintä tehokkaissa rinnakkaisissa videokoodausmenetelmissä on perinteisen
videokoodauksen käyttämien videodatan rinnakkaisuuksien purku sekä
rinnakkaislaskennan synkronoinnin määrän mahdollisimman suuri vähentäminen.
Rinnakkaislaskenta sekä nopeuttaa videokoodausprosessia että parantaa sen
laatua. Rinnakkaistamisen yleistäminen on kuitenkin vaikeaa erilaisten
standardien ja toimivien toteutusten standardoimattomuuden johdosta.

Tuloksista ja tutkimuksesta voidaan vetää kaksi johtopäätöstä. Ensimmäinen on se, että
rinnakkaisuuden tuominen videokoodaukseen on tehokasta ja luultavimmin
tulevaisuudessa välttämätöntä. Toinen on se, että tulevaisuuden
videokoodausstandardeja suunniteltaessa rinnakkaislaskenta täytyy ottaa
huomioon. Täytyy kehittää keino, jolla rinnakkaistaminen onnistuu helposti
erilaisille kiihdytysalustoille.
\end{comment}
%\lipsum[3-4]
%
%Tiivistelmätekstiä tähän (\languagename). Huomaa, että tiivistelmä tehdään %vasta kun koko työ on muuten kirjoitettu.
\end{fiabstract}

%\begin{svabstract}
%  Ett abstrakt hit 
%%(\languagename)
%\end{svabstract}

%\begin{enabstract}
% Here goes the abstract 
%%(\languagename)
%\end{enabstract}
